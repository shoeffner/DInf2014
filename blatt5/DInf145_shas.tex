\documentclass{article}
\usepackage[T1]{fontenc}
\usepackage[utf8]{inputenc}
\usepackage{lmodern}
\usepackage[ngerman]{babel}
\usepackage{amsmath, amssymb}
\usepackage{phonetic} % for reversed D
\usepackage{wasysym}  % for the notes
\usepackage{tikz, tikzsymbols}
\usepackage{xcolor}
\usetikzlibrary{arrows,automata,fit}
\setlength\parindent{0pt}
    
\begin{document}

\begin{center}
  \Large{Informatik \revD: Übungsblatt 5}

  \large{Sebastian Höffner, Andrea Suckro}
\end{center}



\section*{Aufgabe 5.1}
Sie sind gleich mächtig.

Jede Schleife kann durch mehrere Zustände dargestellt werden:

\begin{center}
\begin{tikzpicture}[->, auto, node distance=3cm]
  \node[initial,state]   (Z1)               {$Z_1$};
  \node[state]           (Z2) [right of=Z1] {$Z_2$};
  \node[state,accepting] (Z3) [right of=Z2] {$Z_3$};

  \path (Z1) edge              node {$\epsilon$} (Z2)
        (Z2) edge [loop below] node {$a$}        (Z2)
             edge              node {$\epsilon$} (Z3)
        ;
\end{tikzpicture}
\end{center}

\begin{center}
\begin{tikzpicture}[->, auto, node distance=3cm]
  \node[initial,state]   (Z1)               {$Z_1$};
  \node[state]           (Z2) [right of=Z1] {$Z_2$};
  \node[state]           (Z4) [below of=Z2] {$Z_4$};
  \node[state,accepting] (Z3) [right of=Z2] {$Z_3$};

  \path (Z1) edge              node {$\epsilon$} (Z2)
        (Z2) edge [bend right] node {$a$}        (Z4)
             edge [bend right] node {$a$}        (Z3)
             edge [bend left]  node {$\epsilon$} (Z3)
        (Z4) edge [bend right] node {$a$}        (Z2) 
        ;
\end{tikzpicture}
\end{center}

\section*{Aufgabe 5.2}



\section*{Aufgabe 5.3}



\section*{Aufgabe 5.4}
\begin{enumerate}
	\item bc
	\item cbadbacd
	\item abaaba
	\item cbadc
	\item aababbabaabb
	\item ccaa
	\item cabab
	\item bbbbb
  \item cabab
	\item dccb
  \item acbd
\end{enumerate}

\subsection*{Erklärungen}
Als Tipp ist gegeben bei 9 zu starten.

\bigskip

9 hat zu Beginn \textbf{drei Möglichkeiten}: ein Symbol lesen und es auf das aktuelle Symbol legen, ein $c$ lesen und es auf das aktuelle Symbol lesen oder ein $c$ lesen falls \# auf dem Stack liegt und den Stack unverändert lassen.

Um zu akzeptieren müssen wir letztlich zur Regel kommen, die ein beliebiges Symbol liest wenn \# auf dem Stack liegt, und $\epsilon$ auf den Stack legt.

Wenn wir die erste Möglichkeit nutzen, legen wir Symbole auf den Stack, dadurch verhindern wir, dass wir die dritte Möglichkeit nutzen können, haben aber die Möglichkeit, in die zweite überzugehen.

Wichtig ist, dass die zweite Möglichkeit vom Einlesen und auf den Stack legen eines $d$ gefolgt wird. Da der Kellerautomat durch \textbf{leeren Keller} akzeptiert, ist dies das KO-Kriterium: das $d$ können wir nach den Regeln im ``letzten'' Zustand nicht mehr löschen, der Keller kann also nicht leer werden.

Dadurch bleibt nur die dritte Möglichkeit, $c,\#/\#$. Das heißt, unser Wort muss mit $c$ beginnen.

Nun müssen wir noch vier weitere Zeichen genrieren, bevor wir wieder \# allein auf dem Stack haben. Da es nur eine Regel gibt, die \# auf dem Stack erlaubt, ist das nächste Symbol klar $a$. Darauf folgt eben so logisch $b$, worauf wieder $ab$ folgt, bevor wir \# löschen und akzeptieren.

Das Wort 9 ist also $cabab$.

\bigskip

Als nächstes bietet sich 1 an. Diese Sprache kann nur zwei zweibuchstabige Wörter generieren: $ab$ ($S\rightarrow aA, A\rightarrow C, C\rightarrow b$) und $bc$ ($S\rightarrow bB, B\rightarrow c$). $bc$ passt dank 9.

\bigskip

8 ist das selbe Symbol fünf mal wiederholt. Da wir durch 1 ein $b$ haben, ist 8 $bbbbb$.

\bigskip

Nun bietet sich 2 an, hier haben wir bereits zwei Buchstaben.
Analysiert man den Automaten, stellt man fest, dass die Sprache verschiedene Bausteine mit einem Schleifendurchlauf generieren kann.
\begin{enumerate}
	\item $cd$
  \item $cada$, $cbdb$
  \item $caadaa$, $cabdab$, $cbadba$, $cbbdbb$
  \item $caaadaaa$ $caabdaab$, $cabadaba$, $cbaadbaa$, $cabbdabb$, $cbbadbba$, $cbabdbab$, $cbbbdbbb$
\end{enumerate}
Diese Bausteine können wir beliebig kombinieren, um achtstellige Wörter zu erzeugen. Durch 8 und 9 sind allerdings das fünfte und sechste Zeichen beschränkt: sie können nur $ba$ sein.
Damit fallen alle Wörter aus Fall 4 heraus (fünftes Zeichen $d$), die vierfache Wiederholung von Fall 1, Kombinationen aus Fall 1 und 2 (fünftes oder sechstes Zeichen $c$ oder $d$) sowie alle anderen Kombinationen mit zweimal Fall 2 (fünftes Zeichen $c$).
Lediglich Fall 3 kombiniert mit Fall 1 kann also noch möglich sein. Aus den 8 generierbaren möglichkeiten gibt es nur eine, die $ba$ als fünftes und sechstes Zeichen enthält: $cbadbacd$. Das ist unser Wort.

\bigskip

7 muss auf $ab$ enden und dank 2 ein $a$ an zweiter Stelle haben. Also kann 7 nur $cabab$ sein.

\bigskip

3 sind die bereits bekannten Palindrome mit $a$ und $b$. Da die dritte, fünfte und sechste Stelle gegeben sind ($\_,\_,a,\_,b,a$) können wir das Wort einfach zu $abaaba$ ergänzen.

\bigskip

6 kann mit Wörtern der Länge 4 nur generieren: $c^2ba^{2-1}$ und $c^2a^2$. Da 3 mit a beginnt, geht nur $ccaa$.

\bigskip

10 kann durch einfaches Starten mit $d$ (aus 2) deterministisch gelöst werden und wird $dccb$.

\bigskip

Da 4 nun durch 10 mit $c$ beginnt, kann sie nur $cbadc$ sein.

\bigskip

11 hat nur den Hinweis, dass der dritte Buchstabe $d$ ist. Damit $d$ gelesen weden kann, muss $c$ auf dem Stack liegen, was ein Symbol löscht. Da der Automat durch leeren Keller akzeptiert, muss aber noch ein weiteres Zeichen auf dem Stack verbleiben, damit $d$ das vorletzte Symbol sein kann. Da 5 nur $a$ und $b$ enthalten darf, muss das akzeptierte Zeichen ein $a$ oder $b$ sein. Es gibt nur eine Regel, die das erfüllt: $b,a/\epsilon$. Also muss unter dem $c$ noch ein $a$ auf dem Stack liegen. Die Eingabe muss auf $db$ enden.

\begin{table*}[h]
	\centering
		\begin{tabular}{l|r}
      Eingabe & Stack \\
      \hline
      initial & $\#$ \\
      $a$ & $\# a$ \\
      $c$ & $\# ca$ \\
      $\epsilon$ & $ca$ \\
      $d$ & $a$ \\
      $b$ & $\emptyset$
		\end{tabular}
\end{table*}

\bigskip

Als letztes ist noch 5 zu lösen. Hier haben wir bereits gegeben: \\ $a,\_,b,\_,b,b,\_,b,\_,\_,b,\_$. 

Um das $bb$ erzeugen zu können, müssen wir $aSb$ verwenden, das einzig mögliche passende $a$ für das zweite $b$ ist das allererste. Da nach $a$ ein $\_,b,\_,b$ folgt, das nur durch $S\rightarrow ab$ erzeugt werden kann, muss der erste Teil folgendermaßen struktiert sein: $S \rightarrow aSb \rightarrow aSSb \rightarrow aabSb \rightarrow aababb$.
Der erste Teil ist gefolgt von einem $\_,b$, das durch $S \rightarrow ab$ erzeugt, und durch $S \rightarrow SS$ mit dem bereits vorhandenen ersten Teil verkettet werden kann.

Dieser neue erste Teil kann nun mit den letzten vier Zeichen $\_,\_,b,\_$ erneut durch $S \rightarrow SS$ verbunden werden. Um die letzten Zeichen sinnvoll zu erfüllen, muss für das gegebene $b$ die Regel $S \rightarrow ab$ angewandt werden ($\_,a,b,\_$). Die letzte anzuwendende Regel ist offensichtlich $S \rightarrow aSb$. Damit ist das Wort \linebreak $a--ab-ab--b---ab---a-ab-b = aababbabaabb$.


\section*{Aufgabe 5.5}



\section*{Aufgabe 5.6}
Um zu zeigen, dass alle korrekt geklammerten Klammerfolgen sich als deterministisch kontextfreie Sprache darstellen lassen, malen wir doch einfach mal einen DeterministischenKellerAutomaten:

\begin{center}
\begin{tikzpicture}[->, auto, node distance=3cm]
  \node[initial,state,accepting]   (Z1)               {$Z_1$};
  \node[state]                     (Z2) [right of=Z1] {$Z_2$};
  \path (Z1) edge [bend left] node {(,\#/(\#} (Z2)
        (Z2) edge [bend left] node {$\epsilon,\#/\epsilon$}  (Z1)
             edge [loop right,align=center] node {$(,(/(($ \\ $),(/\epsilon$}  (Z2);
\end{tikzpicture}
\end{center}



\end{document}