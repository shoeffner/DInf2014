\documentclass{article}
\usepackage[T1]{fontenc}
\usepackage[utf8]{inputenc}
\usepackage{lmodern}
\usepackage[ngerman]{babel}
\usepackage{amsmath, amssymb}
\usepackage{array}
\usepackage{phonetic} % for reversed D
\usepackage{wasysym}  % for the notes
\usepackage{tikz, tikzsymbols}
\usepackage{xcolor}
\usetikzlibrary{arrows,automata,fit}
\setlength\parindent{0pt}

\newcommand{\rpt}{%
        \raisebox{.2ex}{:}%
        \raisebox{-.4ex}{\rule{.1ex}{2.5ex}\,\rule{.2ex}{2.5ex}}}
\newcommand{\revrpt}{%
        \raisebox{-.2ex}{\rule{.2ex}{2.5ex}\,\rule{.1ex}{2.5ex}}%
        \raisebox{.2ex}{:}}
    
\begin{document}

\begin{center}
  \Large{Informatik \revD: Übungsblatt 5}

  \large{Sebastian Höffner, Andrea Suckro}
\end{center}



\section*{Aufgabe 6.1}
\begin{center}
\begin{tikzpicture}[->, auto, node distance=2cm]
  \node[initial,state] (Z0)               {$Z_0$};
  \node[state]         (Z1) [right of=Z0] {$Z_1$};
  \node[coordinate]                (C1) [above right of=Z1]{};
  
  \node[state]         (Z2) [above left of=C1] {$Z_2$};
  \node[state]         (Z3) [below right of=C1] {$Z_3$};
  \node[coordinate] (C2) [above right of=Z3]{};
  \node[state]         (Z4) [above of=C2] {$Z_4$};
  \node[state]         (Z5) [right of=Z4] {$Z_5$};
  \node[state]         (Z6) [below of=Z5] {$Z_6$};
  \node[state]         (Z7) [below of=Z4] {$Z_7$};
  \node[state]         (Z8) [below of=Z3] {$Z_8$};
  \node[state]         (Z9) [below of=Z1] {$Z_9$};
  \node[state]         (Z10) [above of=Z2] {$Z_{10}$};
  

  \path (Z0) edge              node {$\epsilon,\$/\#\$$} (Z1)
        (Z1) edge              node {$a,\star\in\Gamma/b\star$} (Z2)
             edge [bend left,pos=0.45] node {$c,a/ba$} (Z3)
        (Z2) edge [bend left,sloped]             node {$a,b/\epsilon$} (Z3)
         edge  [pos=0.3, sloped]            node {$c,b/cc$} (Z3)
         edge node{$\epsilon,\star\in\Gamma/\epsilon$}(Z10)
        (Z3) edge  [bend left]            node {$a,\star\in\Gamma/\epsilon$} (Z1)
        edge      [sloped,pos=0.8]        node {$c,a/cc$} (Z4)
        edge              node {$\circ\in\Sigma,\circ/\circ\circ$} (Z8)
        (Z4) edge              node {$\epsilon,c/cc$} (Z5)
        (Z5) edge              node {$\epsilon,c/ac$} (Z6)
        (Z6) edge              node {$\epsilon,a/ba$} (Z7)
        (Z7) edge              node {$\epsilon,b/ab$} (Z3)
        (Z8) edge              node {$\circ\in\Sigma,\circ/\circ\circ$} (Z9)
        (Z9) edge              node {$\circ\in\Sigma,\circ/\circ\circ$} (Z1)
        (Z10) edge [loop right] node{$\epsilon,\star\in\Gamma/\epsilon$}(Z10)
        ;
\end{tikzpicture}
\end{center}


\section*{Aufgabe 6.2}
\scriptsize
\textbf{Regeln $\forall Z_i \in Z: S \rightarrow R_{(Z_S,\#,Z_i)}$\\}
\begin{tabular}{l}
$S \rightarrow R_{(Z_1,\#,Z_1)}$ \\
$S \rightarrow R_{(Z_1,\#,Z_2)}$ \\
\end{tabular}

\textbf{Regeln für Kanten $a,b/\epsilon$\\}
\begin{tabular}{l}
$R_{(Z_1,b,Z_1)} \rightarrow b$ \\
$R_{(Z_1,\#,Z_1)} \rightarrow b$ \\
$R_{(Z_1,c,Z_1)} \rightarrow c$ \\
\end{tabular}

\textbf{Regeln für Kanten $a,b/c$\\}
\begin{tabular}{l}
$R_{(Z_1,b,Z_1)} \rightarrow bR_{(Z_1,b,Z_1)}$ \\
$R_{(Z_1,b,Z_2)} \rightarrow bR_{(Z_1,b,Z_2)}$ \\
$R_{(Z_2,b,Z_1)} \rightarrow cR_{(Z_2,b,Z_1)}$ \\
$R_{(Z_2,b,Z_2)} \rightarrow cR_{(Z_2,b,Z_2)}$ \\
$R_{(Z_2,b,Z_1)} \rightarrow bR_{(Z_1,c,Z_1)}$ \\
$R_{(Z_2,b,Z_2)} \rightarrow bR_{(Z_1,c,Z_2)}$ \\
\end{tabular}

\textbf{Regeln für Kanten $a,b/cd$\\}
\begin{tabular}{l}
$R_{(Z_1,\#,Z_1)} \rightarrow aR_{(Z_2,b,Z_1)}R_{(Z_1,\#,Z_1)}$ \\
$R_{(Z_1,\#,Z_1)} \rightarrow aR_{(Z_2,b,Z_2)}R_{(Z_2,\#,Z_1)}$ \\
$R_{(Z_1,\#,Z_2)} \rightarrow aR_{(Z_2,b,Z_1)}R_{(Z_1,\#,Z_2)}$ \\
$R_{(Z_1,\#,Z_2)} \rightarrow aR_{(Z_2,b,Z_2)}R_{(Z_2,\#,Z_2)}$ \\
\end{tabular}

\clearpage
\textbf{Regelset mit markierten möglichen Regeln}\\
\begin{tabular}{lll}
\colorbox{green!30}{$S \rightarrow R_{(Z_1,\#,Z_1)}$} & \colorbox{green!30}{$R_{(Z_1,b,Z_1)} \rightarrow bR_{(Z_1,b,Z_1)}$} & \colorbox{green!30}{$R_{(Z_1,\#,Z_1)} \rightarrow aR_{(Z_2,b,Z_1)}R_{(Z_1,\#,Z_1)}$}\\
\colorbox{white!30}{$S \rightarrow R_{(Z_1,\#,Z_2)}$} & \colorbox{white!30}{$R_{(Z_1,b,Z_2)} \rightarrow bR_{(Z_1,b,Z_2)}$} & \colorbox{white!30}{$R_{(Z_1,\#,Z_1)} \rightarrow aR_{(Z_2,b,Z_2)}R_{(Z_2,\#,Z_1)}$}\\
\colorbox{green!30}{$R_{(Z_1,b,Z_1)} \rightarrow b$}  & \colorbox{green!30}{$R_{(Z_2,b,Z_1)} \rightarrow cR_{(Z_2,b,Z_1)}$} & \colorbox{white!30}{$R_{(Z_1,\#,Z_2)} \rightarrow aR_{(Z_2,b,Z_1)}R_{(Z_1,\#,Z_2)}$}\\
\colorbox{green!30}{$R_{(Z_1,\#,Z_1)} \rightarrow b$} & \colorbox{white!30}{$R_{(Z_2,b,Z_2)} \rightarrow cR_{(Z_2,b,Z_2)}$} & \colorbox{white!30}{$R_{(Z_1,\#,Z_2)} \rightarrow aR_{(Z_2,b,Z_2)}R_{(Z_2,\#,Z_2)}$}\\
\colorbox{green!30}{$R_{(Z_1,c,Z_1)} \rightarrow c$}  & \colorbox{green!30}{$R_{(Z_2,b,Z_1)} \rightarrow bR_{(Z_1,c,Z_1)}$} & \\
& \colorbox{white!30}{$R_{(Z_2,b,Z_2)} \rightarrow bR_{(Z_1,c,Z_2)}$} & 
\end{tabular}

Die entstehende Grammatik ist:
\begin{align*}
S &\rightarrow R_{(Z_1,\#,Z_1)} \\
R_{(Z_1,b,Z_1)} &\rightarrow b \\
R_{(Z_1,\#,Z_1)} &\rightarrow b \\
R_{(Z_1,c,Z_1)} &\rightarrow c \\
R_{(Z_1,b,Z_1)} &\rightarrow bR_{(Z_1,b,Z_1)} \\
R_{(Z_2,b,Z_1)} &\rightarrow cR_{(Z_2,b,Z_1)} \\
R_{(Z_2,b,Z_1)} &\rightarrow bR_{(Z_1,c,Z_1)} \\
R_{(Z_1,\#,Z_1)} &\rightarrow aR_{(Z_2,b,Z_1)}R_{(Z_1,\#,Z_1)}
\end{align*}
Gekürzt mit $R_1 = R_{(Z_1,\#,Z_1)}, R_2 = R_{(Z_1,b,Z_1)}, R_3 = R_{(Z_1,c,Z_1)}, R_4 = R_{(Z_2,b,Z_1)}$:
\begin{align*}
S   &\rightarrow R_1               \\
R_1 &\rightarrow aR_4R_1 \ |\ b    \\
R_2 &\rightarrow bR_2    \ |\ b    \\
R_3 &\rightarrow c                 \\
R_4 &\rightarrow cR_4    \ |\ bR_3
\end{align*}
\textit{Anm.: $R_2$ ist nicht erreichbar.}

\normalsize
\section*{Aufgabe 6.3}



\section*{Aufgabe 6.4}



\section*{Aufgabe 6.5}
Sei $L = \left\{\alpha 2 \alpha 2 \alpha | \alpha \in \left\{0,1\right\}^* \right\} \subset \left\{0,1,2\right\}^*$ kontextfrei. Dann existiert gibt es für jedes Wort $z, |z|\geq n$ mit einer Zerlegung $z=uvwxy$ mit $|vx|\geq 1, |uvw|\leq n, \forall i \geq 0: uv^iwx^iy \in L$.

Betrachten wir das Wort $z = \alpha 2 \alpha 2 \alpha$ mit $|\alpha|=n$.

Dann ist $uvw \subseteq \alpha$. 

Egal wie wir das Wort aufpumpen: entweder wir verändern nur einen Teil der $\alpha$s, wodurch diese nicht mehr gleich sind, oder wir pumpen gar etwas mit 2en auf.



\section*{Aufgabe 6.6}
Beim Algorithmus zur Umwandlung NDKA-AdLK in die KF-Grammatik können nur Regeln der folgenden Arten generiert werden:
\begin{enumerate}
	\item $V \rightarrow \epsilon$
  \item $V \rightarrow V$
  \item $V \rightarrow \Sigma$
  \item $V \rightarrow \Sigma \times V$
  \item $V \rightarrow \Sigma \times V \times V$
\end{enumerate}

Die Regeln 1, 3 und 4 sind automatisch in der GNF (1 nach Übungsblatt, 3 und 4 nach Definition).

Für Regeln der Form 2 muss eine Umformung durchgeführt werden, damit sie der Definition der GNF entspricht.

Diese Umformung ist eine Substitution nach folgendem Algorithmus:

Für eine Regel $V \rightarrow U$ mit den Regeln $U_i \rightarrow \left\{W\right\}\cup\left\{\Sigma\right\}\cup\left\{\Sigma\times W\right\}\cup\\\left\{\Sigma \times W \times W\right\}\cup\left\{\epsilon\right\}$ erstelle Regeln $V \rightarrow \left\{W\right\}\cup\left\{\Sigma\right\}\cup\left\{\Sigma\times W\right\}\cup\left\{\Sigma \times W \times W\right\}\cup\left\{\epsilon\right\}$ durch ersetzen der \textit{ersten} Variable mit der Folgeregel $U_i$. Wird die Folgeregel $U_i$ nicht weiter benötigt, entferne sie.
Wiederhole den Vorgang bis nur noch Regeln der Form $V \rightarrow \cup\left\{\Sigma\right\}\left\{\Sigma\times W\right\}\cup\left\{\Sigma \times W \times W\right\}\cup\left\{\epsilon\right\}$ vorhanden sind.

Beispiel:
\begin{align*}
\begin{array}{llll}
Schritt 1 & Schritt 2 & Schritt 3 & Schritt 4 \\
\hline
S \rightarrow sA     & S \rightarrow sA       & S \rightarrow sA            & S \rightarrow sA \\
A \rightarrow aA | B & A \rightarrow aA | bCD & A \rightarrow aA | bDD | cD & A \rightarrow aA | dD | cD \\
B \rightarrow bCD    & C \rightarrow D        & C \rightarrow c             & C \rightarrow c \\
C \rightarrow D      & C \rightarrow c        & D \rightarrow d             & D \rightarrow d \\
C \rightarrow c      & D \rightarrow d        &                             & \\
D \rightarrow d      &                        &                             & \\
\end{array}
\end{align*}


\end{document}
