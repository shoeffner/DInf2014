\documentclass{article}
\usepackage[T1]{fontenc}
\usepackage[utf8]{inputenc}
\usepackage{lmodern}
\usepackage[ngerman]{babel}
\usepackage{amsmath, amssymb}
\usepackage{array}
\usepackage{phonetic} % for reversed D
\usepackage{wasysym}  % for the notes
\usepackage{tikz, tikzsymbols}
\usepackage{xcolor}


\usetikzlibrary{arrows,automata,fit}
\setlength\parindent{0pt}

    
\begin{document}

\begin{center}
  \Large{Informatik \revD: Übungsblatt 8}

  \large{Sebastian Höffner, Andrea Suckro}
\end{center}



\section*{Aufgabe 8.1}
\section*{Aufgabe 8.2}
\section*{Aufgabe 8.3}
\section*{Aufgabe 8.4}
\subsection*{a) GOTO-Programm für $min(x_1,x_2)$}
Die Idee ist hier so lange von beiden Zahlen 1 abzuziehen bis die erste 0 erreicht. Die Zahl die als erstes 0 erreicht ist das Minimum und wird ausgegeben.
\begin{verbatim}
    x4 = x1
    x5 = x2
L1: x4 = x4-1
    x5 = x5-1
    if x4 != 0
      goto L2
    x3 = x1
    halt
L2: if x5 != 0
      goto L1
    x3 = x2
    halt
\end{verbatim}
\section*{Aufgabe 8.5}
Juhu-FreiPunkte $:)$
\section*{Aufgabe 8.6}
Das Kreuzworträtsel ist behindat!


\end{document}
