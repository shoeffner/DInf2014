\documentclass{article}
\usepackage[T1]{fontenc}
\usepackage[utf8]{inputenc}
\usepackage{lmodern}
\usepackage[ngerman]{babel}
\usepackage{amsmath, amssymb}
\usepackage{array}
\usepackage{phonetic} % for reversed D
\usepackage{wasysym}  % for the notes
\usepackage{tikz, tikzsymbols}
\usepackage{xcolor}


\usetikzlibrary{arrows,automata,fit}
\setlength\parindent{0pt}

    
\begin{document}

\begin{center}
  \Large{Informatik \revD: Übungsblatt 8}

  \large{Sebastian Höffner, Andrea Suckro}
\end{center}



\section*{Aufgabe 8.1}
Idee: Da wir beim Start der Maschine schon ganz links also am MSB des Eingabewortes stehen, müssen wir danach nur noch alle danach kommenden Zeichen durch a ersetzen, um auf den Log zu kommen (wir löschen also die erste 1 da $\log_2(10_2) = 1$). Unser Automat behandelt auch noch den Fall, dass das Eingabewort mit beliebig vielen Nullen beginnt. Wenn das Wort nur Nullen enthält soll die Maschine in eine Endlosschleife verfallen, da $\log(0)$ nicht definiert ist.
%Automat für die TM
\begin{center}
\begin{tikzpicture}[->, auto, node distance=2cm]
  \node[state,initial] (S)      {};
  \node[state]         (A) [right of=S] {};
  \node[state]         (B) [right of=A] {};
  \node[state]         (C) [right of=B] {};
 
  \path (S) edge                            node {$1/\square,R$}       (A)
            edge [loop below, align=center] node {$0/\square,R$ \\ $\square/\square,R$} (S) %wir ersetzen die erste 1 
        (A) edge [loop below, align=center] node {$0/a,R$ \\ $1/a,R$}  (A) %wir ersetzen alle Zeichen mit a
            edge                            node {$\square/\square,L$} (B) %am Ende gehen wir nach links
        (B) edge [loop below]               node {$a/a,L$}             (B) %hier gehen wir alle as zurück
            edge                            node {$\square/\square,L$} (C) %damit kommen wir an den Anfang des Wortes
        ;
\end{tikzpicture}
\end{center}

\section*{Aufgabe 8.2}
\section*{Aufgabe 8.3}
\section*{Aufgabe 8.4}
Die Idee ist hier bei beiden Programmen so lange von beiden Zahlen 1 abzuziehen bis die erste 0 erreicht. Die Zahl die als erstes 0 erreicht ist das Minimum und wird ausgegeben.
\subsection*{a) GOTO-Programm für $min(x_1,x_2)$}
%Basti- wenn dir eine bessere Sache für den Code einfällt, dann gerne los :)
\begin{verbatim}
    x4 = x1
    x5 = x2
L1: x4 = x4-1
    x5 = x5-1
    if x4 != 0
      goto L2
    x3 = x1
    halt
L2: if x5 != 0
      goto L1
    x3 = x2
    halt
\end{verbatim}
\subsection*{b) WHILE-Programm für $min(x_1,x_2)$}
\begin{verbatim}
x4 = 1
x5 = x1
x6 = x2
while(x4 != 0){
  x5 = x5-1
  x6 = x6-1
  if(x5 != 0){}
  else{
    x3 = x1
    x4 = 0
  }
  if(x6 != 0){}
  else{
    x3 = x2
    x4 = 0
  }
}
\end{verbatim}
\section*{Aufgabe 8.5}
Juhu-FreiPunkte $:)$
\section*{Aufgabe 8.6}
Das Kreuzworträtsel ist behindat!


\end{document}
