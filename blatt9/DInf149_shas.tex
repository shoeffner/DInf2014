\documentclass{article}
\usepackage[T1]{fontenc}
\usepackage[utf8]{inputenc}
\usepackage{lmodern}
\usepackage[ngerman]{babel}
\usepackage{amsmath, amssymb}
\usepackage{array}
\usepackage{phonetic} % for reversed D
\usepackage{tikz, tikzsymbols}
\usepackage{xcolor}
\usepackage{algorithm2e}


\usetikzlibrary{arrows,automata,fit}
\setlength\parindent{0pt}

    
\begin{document}

\begin{center}
  \Large{Informatik \revD: Übungsblatt 9}

  \large{Sebastian Höffner, Andrea Suckro}
\end{center}

\section*{Aufgabe 9.1}
Ein Beispiel für ein Wort, das weder selbstidentifizierend noch nicht selbstidentifizierend ist, ist das Wort nichtselbstidentifizierend selbst.
\begin{itemize}
  \item[1:] \textbf{nichtselbstidentifizierend ist nichtselbstidentifizierend} \\
    Wenn nichtselbstidentifizierend sich selbst nicht beschreibt, dann ist es Teil der Menge der nicht selbstidentifizierenden Wörter. Da die Wörter dieser Menge sich selbst nicht identifizieren, sind sie nichtselbstidentifizierend. Dies ist ein Widerspruch.
  \item[2:] \textbf{nichtselbstidentifizierend ist selbstidentifizierend} \\
    Wenn nichtselbstidentifizierend sich selbst beschreibt, dann ist es Teil der Menge der selbstidentifizierenden Wörter. Die Bedeutung von nichtselbstidentifizierend ist jedoch, dass ein Wort sich nicht selbst beschreibt, wodurch wir erneut einen Widerspruch haben.
\end{itemize}

\bigskip
\textit{nichtselbstidentifizierend} führt in beiden Betrachtungsweisen zum Widerspruch. Wir haben also ein Wort, dass weder zur einen, noch zur anderen Kategorie passt.

\section*{Aufgabe 9.2}
Wir prüfen für alle möglichen Eingaben, ob der Zustand $Z_i$ erreicht wird. Sobald $Z_i$ erreicht wird, gibt der Algorithmus 1 zurück. So lange wir keine gültige Eingabe finden, terminiert der Algorithmus nicht, d.h. wir können keine Aussage dazu treffen.

\begin{algorithm}[ht]
  \KwIn{Input X}
  \KwOut{1 bei Erfolg}
  \For{n = 0, 1, 2, \dots}{
    \For{$\forall w \in \Sigma^n$} {
      \If{TM reaches $Z_i$ for input w}{\Return{1}}
    }
  }
  \caption{Pseudocode zu 9.2}
\end{algorithm}

\section*{Aufgabe 9.3}
Wir können ein Programm $\mathcal{P}$ nutzen, welches terminiert, sobald 4 erreicht wird. Dieses Programm können wir mit $\mathcal{A}(\mathcal{P})$ prüfen: wenn $\mathcal{A}(\mathcal{P}) = 1$, dann ist die Collatzvermutung wahr, sonst nicht.
\begin{algorithm}[ht]
  \KwIn{Input n}
  \KwOut{1 bei Erreichen von 4}
  \While{$n \neq 4$}{
    n := C(n)\;
  }
  \Return{1}
  \caption{Pseudocode zu Programm $\mathcal{P}$ von 9.3}
\end{algorithm}


\section*{Aufgabe 9.4}
Um zu zeigen, dass die Turingmaschine auf keiner möglichen Eingabe hält, muss ich für alle Eingaben zeigen, dass sie nicht hält.

Die Prüfung für jede Eingabe, ob sie hält oder nicht, ist das klassische Halteproblem. Dieses Problem ist nicht entscheidbar, und damit ist auch die Frage, ob die Turingmaschine auf keiner möglichen Eingabe hält, unentscheidbar.

\bigskip

Problematisch wird z.B. folgendes Programm $\mathcal{P}$, bei dem nicht entscheidbar ist, ob es hält oder nicht. Hält es nicht, d.h. wird die \texttt{if}-Bedingung true, so hält es - hält es, d.h. die \texttt{if}-Bedingung ist false, hält es dank der Endlosschleife nicht.
\begin{algorithm}[ht]
  \If{hältNicht($\mathcal{P}$)}{
    \Return{1}
  }
  \Else{\While{true}{
    NOOP;
  }}
  \caption{Pseudocode zu Programm $\mathcal{P}$ von 9.4}
\end{algorithm}
Das Problem ist also \textbf{nicht entscheidbar} für jede Turingmaschine.




\section*{Aufgabe 9.5}
\subsection*{a)}
Diese Instanz hat eine Lösung und zwar: \textsc{wein hier und bier dort}.
\subsection*{b)}
Für diese Instanz lässt sich keine Lösung finden, da sich eins von diesen beiden Mustern (abhängig vom Startstein) für immer wiederholt: $p(am)^*$ oder $p(aamm)^*$
\bigskip
\emph{Alternative:} Betrachten wir die Steine ebenfalls als Tupel $(x_i,y_i)$. Dann kann es verschiedene Fälle geben, wie $|x_i|$ und $|y_i|$ zusammenhängen.

In den einfachsten entscheidbaren Fällen gilt entweder $\forall i: |x_i| < |y_i|$ oder $\forall i: |x_i| > |y_i|$. Dann ist es \emph{nicht} möglich, Lösungen zu finden, da immer eine "`Seite"' länger wird, als die andere.

Der zweite Fall ist, dass es entweder gilt $\forall i: |x_i| \leq |y_i|$ oder $\forall i: |x_i| \geq |y_i|$. Diese Fälle können wir vereinfachen, da für die Lösung nur die Fälle $\forall i: |x_i| = |y_i|$ interessant sind, ansonsten haben wir das selbe Problem wie in den ersten Fällen: Eine "`Seite"' würde länger, als die andere. Das heißt, wir können alle Tupel $\forall i: |x_i| \neq |y_i|$ entfernen und auf den verbliebenen Tupeln nach der Lösung suchen. Diese Tupel fallen in den nächsten Fall:

Damit wir eine Folge als Lösung finden können, muss in mindestens einer beliebigen Kombination $\sum_j^n |x_{i_j}| = \sum_j^n |y_{i_j}|$ gegeben sein. Dazu können $|x_i|$ und $|y_i|$ beliebig zueinander stehen, da wir sie beliebig kombinieren können. Auf dem nach diesen Fallunterscheidungen noch sinnvollen Tupeln kann nun nach einer Lösung gesucht werden.

In unserem konkreten Beispiel können wir nach der Argumentation oben drei Tupel ($(p,aam),(p,paa),(pa,pam)$) eliminieren, es bleiben übrig $(a,m)$ und $(m,a)$. 

Startet man mit einem Tupel, kann man nun das selbe oder das andere der beiden Tupel davor oder dahinter hängen, in beiden Fällen ist es nicht möglich, in beiden Wörtern das selbe Wort zu erzeugen. Das liegt daran, dass für jedes $a$ auf der oberen Hälfte ein $m$ auf der unteren erzeugt wird und umgekehrt, wir können aber niemals beide Wörter mit dem gleichen Buchstaben beginnen lassen.



\section*{Aufgabe 9.6}

\subsection*{a)}
Der Rückgabewert ist immer korrekt, weil entweder eine Verkürzung gefunden wurde (z.B. wird \texttt{aaaa} zu \texttt{4:a} statt der trivialen Lösung \texttt{1:aaaa}), dessen Länge ausgegeben wird, oder aber die Länge des Trivialfalls ($|w|:w$) zurückgegeben wird. Diese berechnet sich aus $|w|+\lfloor \log_10 |w| \rfloor + 2$. Deshalb ist auch $\mathcal{K}(w)\leq |w|+l+1$ korrekt.

\subsection*{b)}
Ja, für jede endliche Eingabe $w$ terminiert $\mathcal{K}(w)$. Dies ist garantiert, weil PURL-Programme regulär und endlich sind.

Die Umformung von PURL zu regulären Sprachen, auf denen $a^n$ durch "`wiederhole $a$ $n$ Mal"' definiert ist, ist:
\begin{itemize}\itemsep0em
	\item $\epsilon \rightarrow \epsilon$
  \item $AB \rightarrow AB$
  \item $\phi\!:\!\varrho; \varrho \in \Sigma^c, \phi=\mathbb{D}(c) \rightarrow \varrho^c$
  \item $\phi(A); \phi=\mathbb{D}(c) \rightarrow (A)^c$
\end{itemize}
Die Programme sind endlich, weil sie niemals endlose Wiederholungen ermöglichen, es gibt keine Kleenesche Hülle.

D.h. für jedes endliche Eingabe hat der Algorithmus endliche Laufzeit, die jedoch schnell sehr groß wird. $\mathcal{K}(w)$ kann also sehr lange brauchen, terminiert aber immer.

\subsection*{c)}
Wir berechnen lediglich das kürzeste PURL-Programm, nicht die Kolmogorov-Komplexität - letztere ist nämlich nur für Turing-vollständige Sprachen definiert.

\end{document}