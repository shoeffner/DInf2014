\documentclass{article}
\usepackage[T1]{fontenc}
\usepackage[ansinew ]{inputenc}
\usepackage{amsmath}
\usepackage{tikz}
\usepackage{tikzsymbols}
\usepackage{lmodern}
\usetikzlibrary{arrows,automata}
\setlength\parindent{0pt}

\begin{document}

\begin{center}
  \Large{Informatik D - �bungsblatt 3}

  \large{Sebastian H�ffner, Andrea Suckro}
\end{center}

\section{Aufgabe 3.1}
Sei $L \subseteq \Sigma^{*}$ eine endliche Sprache, d.h. eine Sprache mit endlicher Anzahl W�rtern $\omega \in L$ endlicher L�nge. Dann k�nnen wir eine Grammatik definieren, die individuell alle W�rter aus $L$ generiert.
\begin{align*}
S\ \rightarrow\ \omega_i\ |\ \omega_{i+1}\ |\ ...
\end{align*}
Diese Grammatik kann in eine regul�re Grammatik �berf�hrt werden, indem jedes Wort $\omega_i$ als Verkettung von Regeln dargestellt wird.
\begin{align*}
S\ &\rightarrow\ O_{i,0}\ |\ O_{i+1,0}\ |\ ...\\
O_{i,0}\ &\rightarrow\ \omega_{i,0} O_{i,1} \\
O_{i,1}\ &\rightarrow\ \omega_{i,1} O_{i,2} \\
...
\end{align*}
In der Vorlesung wurde bereits gezeigt, dass regul�re Grammatiken in regul�re Ausdr�cke und endliche Automaten �berf�hrt werden k�nnen, also ist die endliche Sprache ebenfalls regul�r.


\section{Aufgabe 3.2}


\section{Aufgabe 3.3}


\section{Aufgabe 3.4}


\section{Aufgabe 3.5}


\end{document}