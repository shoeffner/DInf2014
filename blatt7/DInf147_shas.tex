\documentclass{article}
\usepackage[T1]{fontenc}
\usepackage[utf8]{inputenc}
\usepackage{lmodern}
\usepackage[ngerman]{babel}
\usepackage{amsmath, amssymb}
\usepackage{array}
\usepackage{phonetic} % for reversed D
\usepackage{wasysym}  % for the notes
\usepackage{tikz, tikzsymbols}
\usepackage{xcolor}
\usetikzlibrary{arrows,automata,fit}
\setlength\parindent{0pt}

\newcommand{\rpt}{%
        \raisebox{.2ex}{:}%
        \raisebox{-.4ex}{\rule{.1ex}{2.5ex}\,\rule{.2ex}{2.5ex}}}
\newcommand{\revrpt}{%
        \raisebox{-.2ex}{\rule{.2ex}{2.5ex}\,\rule{.1ex}{2.5ex}}%
        \raisebox{.2ex}{:}}
    
\begin{document}

\begin{center}
  \Large{Informatik \revD: Übungsblatt 7}

  \large{Sebastian Höffner, Andrea Suckro}
\end{center}



\section*{Aufgabe 7.1}
\section*{Aufgabe 7.2}
\section*{Aufgabe 7.3}
\subsection*{Leerheitsproblem}
\begin{center}
Initial:\\
\begin{tabular}{ll}
$S \rightarrow E | ABC$               & $A \rightarrow bDC | E | S$ \\
$B \rightarrow acDC | C$              & $C \rightarrow ab | Ja | IdA$ \\
$D \rightarrow deF | DeF | dEF | DEF$ & $E \rightarrow abba | aBBa | AbbA$ \\
$F \rightarrow JcJ | GJ | bG$         & $G \rightarrow F | IG$ \\
$H \rightarrow SA | Hcc$              & $I \rightarrow c | cIc | dF$
\end{tabular}\\
Erster Schritt:\\
\begin{tabular}{ll}
$S \rightarrow {\color{red}E} | AB{\color{red}C}$               & $A \rightarrow bD{\color{red}C} | {\color{red}E} | S$ \\
$B \rightarrow acD{\color{red}C} | {\color{red}C}$              & ${\color{red}C} \rightarrow ab | Ja | {\color{red}I}dA$ \\
$D \rightarrow deF | DeF | d{\color{red}E}F | D{\color{red}E}F$ & ${\color{red}E} \rightarrow abba | aBBa | AbbA$ \\
$F \rightarrow JcJ | GJ | bG$                                   & $G \rightarrow F | {\color{red}I}G$ \\
$H \rightarrow SA | Hcc$                                        & ${\color{red}I} \rightarrow c | c{\color{red}I}c | dF$
\end{tabular}\\
Zweiter Schritt:\\
\begin{tabular}{ll}
${\color{red}S} \rightarrow {\color{red}E} | {\color{red}ABC}$  & ${\color{red}A} \rightarrow bD{\color{red}C} | {\color{red}E} | {\color{red}S}$ \\
${\color{red}B} \rightarrow acD{\color{red}C} | {\color{red}C}$ & ${\color{red}C} \rightarrow ab | Ja | {\color{red}I}d{\color{red}A}$ \\
$D \rightarrow deF | DeF | d{\color{red}E}F | D{\color{red}E}F$ & ${\color{red}E} \rightarrow abba | a{\color{red}BB}a | {\color{red}A}bb{\color{red}A}$ \\
$F \rightarrow JcJ | GJ | bG$                                   & $G \rightarrow F | {\color{red}I}G$ \\
$H \rightarrow {\color{red}SA} | Hcc$                           & ${\color{red}I} \rightarrow c | c{\color{red}I}c | dF$
\end{tabular}
\end{center}
Bereits im zweiten Schritt wird $S$ markiert, also ist $L(G) \neq \emptyset$.

\subsection*{Endlichkeitsproblem}
Wir führen den Algorithmus von oben vorerst zu Ende.
\begin{center}
Dritter Schritt:\\
\begin{tabular}{ll}
${\color{red}S} \rightarrow {\color{red}E} | {\color{red}ABC}$  & ${\color{red}A} \rightarrow bD{\color{red}C} | {\color{red}E} | {\color{red}S}$ \\
${\color{red}B} \rightarrow acD{\color{red}C} | {\color{red}C}$ & ${\color{red}C} \rightarrow ab | Ja | {\color{red}I}d{\color{red}A}$ \\
$D \rightarrow deF | DeF | d{\color{red}E}F | D{\color{red}E}F$ & ${\color{red}E} \rightarrow abba | a{\color{red}BB}a | {\color{red}A}bb{\color{red}A}$ \\
$F \rightarrow JcJ | GJ | bG$                                   & $G \rightarrow F | {\color{red}I}G$ \\
${\color{red}H} \rightarrow {\color{red}SA} | {\color{red}H}cc$ & ${\color{red}I} \rightarrow c | c{\color{red}I}c | dF$
\end{tabular}
\end{center}
Weiter lässt sich nichts markieren, d.h. wir können die Grammatik nun auf die markierten Regeln reduzieren.
\begin{center}
\begin{tabular}{ll}
$S \rightarrow E | ABC$            & $A \rightarrow E | S$ \\
$B \rightarrow C$                  & $C \rightarrow ab | IdA$ \\
$E \rightarrow abba | aBBa | AbbA$ & $H \rightarrow SA | Hcc$ \\
$I \rightarrow c | cIc$            & \\
\end{tabular}
\end{center}
Als nächstes reduzieren wir die Grammatik auf erreichbare Regeln.
\begin{center}
\begin{tabular}{ll}
${\color{red}S} \rightarrow {\color{red}E} | {\color{red}ABC}$                         & ${\color{red}A} \rightarrow {\color{red}E} | {\color{red}S}$ \\
${\color{red}B} \rightarrow {\color{red}C}$                                            & ${\color{red}C} \rightarrow ab | {\color{red}IdA}$ \\
${\color{red}E} \rightarrow abba | a{\color{red}BB}a | {\color{red}A}bb{\color{red}A}$ & $H \rightarrow SA | Hcc$ \\
${\color{red}I} \rightarrow c | c{\color{red}I}c$                                      & \\
\end{tabular}
\end{center}
Übrig bleiben:
\begin{center}
\begin{tabular}{ll}
$S \rightarrow E | ABC$            & $A \rightarrow E | S$ \\
$B \rightarrow C$                  & $C \rightarrow ab | IdA$ \\
$E \rightarrow abba | aBBa | AbbA$ & $I \rightarrow c | cIc$ \\
\end{tabular}
\end{center}
Die Regeln $S \rightarrow E$, $A \rightarrow E$, $A \rightarrow S$ und $B \rightarrow C$ müssen eliminiert werden. Die transformierte Grammatik sieht z.B. so aus:
\begin{center}
\begin{tabular}{ll}
$S \rightarrow abba | aBBa | AbbA | ABab | ABIdA$ & $A \rightarrow abba | aBBa | AbbA | abba | aBBa | AbbA | ABab | ABIdA$ \\
$B \rightarrow ab | IdA$                          & $I \rightarrow c | cIc$ \\
\end{tabular}
\end{center}

Wir erstellen einen Hilfsgraph auf Grundlage der reduzierten und vereinfachten Grammatik.
\begin{center}
\begin{tikzpicture}[->, auto, node distance=2cm]
  \node[state] (S)              {$S$};
  \node[state] (A) [right of=S] {$A$};
  \node[state] (B) [below of=A] {$B$};
  \node[state] (I) [left of=B]  {$I$};

  \path (S) edge node {} (B)
            edge node {} (A)
            edge node {} (I)
        (A) edge [bend left] node {} (B)
        (B) edge [bend left] node {} (A)
            edge node {} (I)
        ;
\end{tikzpicture}
\end{center}

Es ist ein gerichteter Kreis $A \leftrightarrow B$ enthalten. Somit ist $|L(G)| = \infty$.

\section*{Aufgabe 7.4}
\section*{Aufgabe 7.5}
\section*{Aufgabe 7.6}

\end{document}
