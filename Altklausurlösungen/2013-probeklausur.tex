\documentclass{article}
\usepackage[T1]{fontenc}
\usepackage[utf8]{inputenc}
\usepackage{amsmath,amssymb}
\usepackage{tikz}
\usepackage{tikzsymbols}
\usepackage{tabularx}
\usepackage{lmodern}
\usepackage{xcolor}
\usepackage{algorithm2e}
\usepackage{listings}
\usepackage{titlesec}
\usepackage{hyperref}
\usepackage[ngerman]{babel}
\titlelabel{\thetitle.\ Aufgabe}
\usetikzlibrary{arrows,automata}
\setlength\parindent{0pt}

\begin{document}

\begin{center}
  \Large{Informatik D -- Probeklausur 2013}

  \large{Sebastian Höffner, Andrea Suckro}
\end{center}

\section{}%1
\begin{itemize}
	\item reguläre Sprachen
  \item rekursiv aufzählbare Sprachen
  \item Deterministisch Kontextfreie Sprachen
\end{itemize}

\section{}%2
Nach rechts auffüllen beginnend bei\dots
\begin{itemize}
	\item \dots kontextfrei
	\item \dots regulär $(a|b)^*|(cc|dd)$
	\item \dots deterministisch kontextfrei
	\item \dots kontextsensitiv
\end{itemize}

\section{}%3
\begin{align*}
S &\rightarrow a | b | aS | bS | aA | bA \\
A &\rightarrow c | cB | C \\
B &\rightarrow a \\
C &\rightarrow d | dB
\end{align*}

\section{}%4
\begin{tikzpicture}[->, auto, node distance=3cm]
	\node[initial,state,accepting]   (A) {};
  \node[state] (B) [above of = A] {};
  \node[state] (C) [right of = B] {};
  \node[state] (D) [right of = C] {};
  \node[state] (E) [below of = D] {};
  \node[state] (F) [right of = D] {};
  \node[state,accepting] (G) [left of=E] {};
  \path (A) edge [] node {$a,\#/\#$} (B)
        (B) edge [bend left, above] node {$b,\star\!\in\!\Sigma/\$\star$} (C)
        (C) edge [bend left, below] node {$a,\star\!\in\!\Sigma/\star$} (B)
            edge [] node {$c,\$/\epsilon$} (D)
        (D) edge [left] node {$c,\star\!\in\!\Sigma/\star$} (E)            
        (E) edge [right] node {$c,\$/\$$} (F)
            edge [below] node {$c,\#/\#$} (G)
        (F) edge [] node {$c,\$/\epsilon$} (D);
\end{tikzpicture}

\section{}%5
\subsection{}%a
\begin{itemize}
	\item \fbox{Ja}, \emph{jede kontextfreie Sprache erfüllt das PL-KF}. 
  \item \fbox{Ja}, \emph{es gibt nicht-kontextfreie Sprachen, die das PL-KF erfüllen}. Deshalb können wir nur zeigen, dass Sprachen nicht kontextfrei sind, nicht aber, ob dass sie es sicher sind.
  \item \fbox{Ja}, \emph{es gibt deterministische kontextfreie Sprachen, die das PL-KF erfüllen}. Sogar reguläre.
  \item \emph{Beim PL-KF wird ein Wort in \fbox{5} Teilworte zerlegt.}
\end{itemize}

\subsection{}%b
Wir nehmen an, $L=\{a^ib^ic^{2i} | i \in \mathbb{N}\}$ sei kontextfrei. Dann gibt es eine Zahl $n = n(L)$, für die für jedes Wort $z \in L$ eine Zerlegung $uvwxy$ existiert, mit $|z|\geq n$, $|vx|\geq 1$, $|vwx| \leq n$, sodass $uv^*wx^*y \in L$.

Wir betrachten das Wort $z = a^nb^nc^{2n}$. Wir betrachten nun alle verschiedene Möglichkeiten, $z$ zu zerlegen.
\begin{enumerate}
	\item $v = a^l, x = a^k$ mit $l + k \leq n$, dann ist $w = a^{n-l-k}$. Egal wie wir $v$ und $x$ aufpumpen, die Anzahl $a$s wird nicht mehr zur Anzahl der $b$s passen. Widerspruch.
  \item $v = b^l, x = b^k$ mit $l + k \leq n$ ist analog zu 1. Widerspruch.
  \item $v = c^l, x = c^k$ mit $l + k \leq n$ ist analog zu 1. Widerspruch.
  \item $v = a^l, x = b^k$ mit $l + k \leq n$: Die Anzahl $a$s oder $b$s ist nach aufpumpen nicht mehr passend zur Anzahl $c$s. Widerspruch.
  \item $v = b^l, x = c^l$ analog zu 4. Widerspruch.
\end{enumerate}
Die Sprache $L$ ist also nicht kontextfrei.

\section{}%6
\begin{tikzpicture}[->, auto, node distance=3cm]
	\node[initial,state]   (A) {};
  \node[state] (B) [right of = A] {};
  \node[state] (C) [below right of = B] {};
  \node[state] (D) [left of = C] {};
  \node[state] (E) [below left of = C] {};
  \node[state] (F) [left of = E] {};
  \path (A) edge [] node {$\star\!\in\!\Sigma/\star,R$} (B)
            edge [] node {$\square/0,S$} (D)
        (B) edge [] node {$\star\!\in\!\Sigma/\star,R$} (C)
            edge [] node {$\square/0,S$} (D)
        (C) edge [] node {$\square/0,S$} (D)
            edge [] node {$\star\!\in\!\Sigma/\star,L$} (E)
        (E) edge [] node {$\star\!\in\!\Sigma/\star,L$} (F);
\end{tikzpicture}

\section{}%7
\begin{enumerate}
	\item ja (CYK algorithm)
	\item nein (DKF nur abgeschlossen über Komplement)
	\item ja? (ich glaube so, aber nicht andersrum (s.u.))
	\item ja? (ich glaube so, aber nicht andersrum (s.n.))
	\item nein 
	\item nein 
	\item nein (denke semi-entscheidbar?)
	\item ja (siehe Q\&A session)
	\item nein (der Klassiker)
\end{enumerate}

\section{}%8
\begin{lstlisting}[mathescape]
    $x_3$ := 0
    $x_4$ := $x_1$
    $x_5$ := $x_2$
L1: if($x_4 \neq 0$) goto L2
    if($x_5 \neq 0$) goto L3
    halt
L2: $x_4 := x_4 - 1$
    $x_3 := x_3 + 2$
    goto L1
L3: $x_5 := x_5 - 1$
    $x_3 := x_3 + 1$ 
    goto L1
\end{lstlisting}

\section{}%9
\begin{center}
\begin{tikzpicture}
    % PSPACE area
    \draw (0, 1.5em) node[rectangle,fill=green!10,text width=12em,align=center,minimum height=10em] (PSPACEFILL) {};
    % PSPACE text
    \draw (0, 5em) node[text width=12em,align=center] (PSPACE) {PSPACE};
    % NP
    \draw (0, 0) node[rectangle,fill=gray!20,text width=11em,align=center,minimum height=7em] (NP) {NP};
    % NP complete
    \draw (0, 0.75) node[rectangle,fill=red!20,text width=10em,align=center] (NPc) {NP-vollständig};
    % P
    \draw (0, -0.75) node[rectangle,fill=red!20, text width=10em,align=center] (P) {P};
    % np hard border
    \draw[dashed,thick,red] (-7em,7em) -- (-7em,1em) -- (14em,1em) -- (14em,7em);
    % np hard label
    \draw (10em,4em) node[red] (NPh) {NP-schwer};
\end{tikzpicture}
\end{center}
\begin{enumerate}
	\item[P:] Ist Array sortiert?, Ist Sat-Belegung korrekt?
  \item[NP$\backslash$P:] BinPacking, SubsetSum
\end{enumerate}

\section{}%10
\begin{itemize}
	\item Zeige, dass $\mathcal{X}\in NP$ (z.B. durch Finden eines deterministisch polynomiellen Zeugens).
  \item Reduziere in deterministisch polynomieller Zeit eine zufällige, beliebige Instanz $I$ eines passenden Problems $\mathcal{Y}_i$ auf eine frei wählbare Instanz $I'$ aus $\mathcal{X}$.
  \item Prüfe, ob die Reduktion korrekt ist dadurch, dass $I \text{erfüllbar} \Leftrightarrow I' \text{erfüllbar}$ gilt, d.h. zeige beide Richtungen: I erfüllbar gdw I' erfüllbar und I' erfüllbar gdw I erfüllbar.
\end{itemize}
\emph{Ja}, wir können den Beweis genau so durchführen, da wir für $\mathcal{X}$ zeigen, dass es in NP liegt und durch die Reduktion, dass es NP-schwer ist.

\end{document}